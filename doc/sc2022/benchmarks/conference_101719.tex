\documentclass[conference]{IEEEtran}
\IEEEoverridecommandlockouts
% The preceding line is only needed to identify funding in the first footnote. If that is unneeded, please comment it out.
\usepackage{cite}
\usepackage{amsmath,amssymb,amsfonts}
\usepackage{algorithmic}
\usepackage{graphicx}
\usepackage{textcomp}
\usepackage{xcolor}
\def\BibTeX{{\rm B\kern-.05em{\sc i\kern-.025em b}\kern-.08em
T\kern-.1667em\lower.7ex\hbox{E}\kern-.125emX}}
\begin{document}

    \title{SAT-X Unsolved Problems Benchmarks\\
%{\footnotesize \textsuperscript{*}Note: Sub-titles are not captured in Xplore and should not be used}
%\thanks{Thanks to all supporters of http://www.peqnp.com projects.}
    }

    \author{\IEEEauthorblockN{1\textsuperscript{st} Oscar Riveros}
    Santiago, Chile \\
    oscar.riveros@gmail.com
    }

    \maketitle

    \begin{abstract}
        A set of unsolved and recent solved problems benchmarks for SAT Competition 2022, this problemas are created with SAT-X python library https://github.com/maxtuno/SATX, that is a solver free advanced version of PEQNP python library presented on SAT Competition 2021\cite{b1}.
    \end{abstract}


    \section{Introduction}
    To solve some problems like the sum of three cubes for 33, 42, or 3 (large representation), a large amount of computing power was required, then is very interesting try to solve this with SAT Solvers.


    \section{Methods}

    \subsection{Descriptions of problems}

    \subsubsection{3D perfect Euler bricks}

    \begin{itemize}
        \item $a^2 + b^2 = p^2$
        \item $a^2 + c^2 = q^2$
        \item $b^2 + c^2 = r^2$
        \item $a^2 + b^2 + c^2 = s^2$
    \end{itemize}

    \subsubsection{4D Euler bricks}

    \begin{itemize}
        \item $a^2 + b^2 = p^2$
        \item $a^2 + c^2 = q^2$
        \item $b^2 + c^2 = r^2$
        \item $a^2 + d^2 = s^2$
        \item $b^2 + d^2 = t^2$
        \item $c^2 + d^2 = u^2$
    \end{itemize}

    \subsubsection{4D perfect Euler bricks}

    \begin{itemize}
        \item $a^2 + b^2 = p^2$
        \item $a^2 + c^2 = q^2$
        \item $b^2 + c^2 = r^2$
        \item $a^2 + d^2 = s^2$
        \item $b^2 + d^2 = t^2$
        \item $c^2 + d^2 = u^2$
        \item $a^2 + b^2 + c^2 + d^2 = v^2$
    \end{itemize}

    \subsubsection{Brocard's problem}

    $n!+1 = m^2, n > 7$, This problem is presented on $16$, $32$ bits format.

    \subsubsection{H31 - The smallest (in H) open equation \cite{b2}}

    $y(x^3 - y) = z^3 + 3$, This problem is presented on $80$ and $128$, $256$ bits format.

    \subsubsection{Sum of three cubes}

    $x^3 + y^3 + z^3 == k, k \in \{3, 33, 42, 165, 906, 114, 390, 579, 627, 633, 732, 921, 975\}$, This problem is for known solutions for $3, 33, 42$, to compare with large scale computation needed to solve, and search for unknown solutions over actual solutions. For un solved values is presented on $3 * 80$ and $3 * 128$ bits format.

    \begin{thebibliography}{00}
        \bibitem{b1} Balyo , T , Froleyks , N , Heule , M , Iser , M , Järvisalo , M & Suda , M (eds) 2021 , Proceedings of SAT Competition 2021 : Solver and Benchmark Descriptions . Department of Computer Science Report Series B , vol. B-2021-1 , Department of Computer Science, University of Helsinki , Helsinki .
        \bibitem{b2} Grechuk, B. (2021). Diophantine equations: a systematic approach.
    \end{thebibliography}
    \vspace{12pt}

\end{document}
